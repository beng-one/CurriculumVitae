%
% CV corrigé avec photo à gauche et infos à droite
%

\documentclass[a4paper,11pt]{article}

% Packages
\usepackage{latexsym}
\usepackage{xcolor}
\usepackage{float}
\usepackage{ragged2e}
\usepackage[empty]{fullpage}
\usepackage{wrapfig}
\usepackage{lipsum}
\usepackage{tabularx}
\usepackage{titlesec}
\usepackage{geometry}
\usepackage{marvosym}
\usepackage{verbatim}
\usepackage{enumitem}
\usepackage{fancyhdr}
\usepackage{multicol}
\usepackage{graphicx}
\usepackage{cfr-lm}
\usepackage[T1]{fontenc}
\usepackage{fontawesome5}
\usepackage[hidelinks]{hyperref}

% Couleur principale
\definecolor{darkblue}{RGB}{0,0,139}

% Mise en page
\geometry{left=1.4cm, top=0.8cm, right=1.2cm, bottom=1cm}
\setlength{\footskip}{5pt}
\setlength{\multicolsep}{0pt}
\urlstyle{same}
\pagestyle{fancy}
\fancyhf{}
\fancyfoot{}
\renewcommand{\headrulewidth}{0pt}
\renewcommand{\footrulewidth}{0pt}

% Sections
\titleformat{\section}{
  \vspace{-4pt}\scshape\raggedright\large
}{}{0em}{}[\color{black}\titlerule \vspace{-7pt}]

% Hyperliens
\hypersetup{
    colorlinks=true,
    linkcolor=darkblue,
    urlcolor=darkblue,
}

% Boîtes de section
\usepackage[most]{tcolorbox}
\tcbset{
    frame code={},
    center title,
    left=0pt,
    right=0pt,
    top=0pt,
    bottom=0pt,
    colback=gray!20,
    colframe=white,
    width=\dimexpr\textwidth\relax,
    enlarge left by=-2mm,
    boxsep=4pt,
    arc=0pt,outer arc=0pt,
}

% Commandes utiles
\newcommand{\resumeSubheading}[4]{
\vspace{0.5mm}\item
    \begin{tabular*}{0.98\textwidth}[t]{l@{\extracolsep{\fill}}r}
        \textbf{#1} & \textit{\footnotesize{#4}} \\
        \textit{\footnotesize{#3}} &  \footnotesize{#2}\\
    \end{tabular*}
    \vspace{-2.4mm}
}

\newcommand{\resumeItemListStart}{\begin{itemize}[leftmargin=*,labelsep=1mm,itemsep=0.5mm]}
\newcommand{\resumeItemListEnd}{\end{itemize}\vspace{-2mm}}
\newcommand{\resumeSubHeadingListStart}{\begin{itemize}[leftmargin=*,labelsep=1mm]}
\newcommand{\resumeSubHeadingListEnd}{\end{itemize}\vspace{2mm}}

%----------------------------------
\begin{document}

%------------------- EN-TÊTE ----------------------%
\begin{minipage}[t]{0.23\textwidth}
    \vspace{0pt}
    \begin{flushleft}
        % Photo à gauche (très petite)
        \includegraphics[width=3cm,height=4cm,keepaspectratio,clip]{cv_Photo_pro.png}
    \end{flushleft}
\end{minipage}
\begin{minipage}[t]{0.73\textwidth}
    \vspace{0pt}
    \raggedleft
    {\Large\textbf{Lajoie BENGONE AKOU}} \\[-1pt]
%    \large
    \small
    \faEnvelope \hspace{2pt} \href{mailto:Bengonelajoie@gmail.com}{Bengonelajoie@gmail.com} \\[-1pt]
    \faPhone \hspace{2pt} +33 6 19 95 30 53 \\[-1pt]
    \faMapMarker* \hspace{2pt} Île-de-France \\[-1pt]
    \faLinkedin \hspace{2pt} \href{https://www.linkedin.com/in/lajoie-bengone-akou/}{linkedin.com/in/lajoie-bengone-akou} \\[-1pt]
    \faGithub \hspace{3pt} \href{https://github.com/beng-one}{github.com/beng-one}\\[-1]
\end{minipage}
% ------------------- PROFIL -------------------------------
\section{\textbf{PROFIL}}
\vspace{-0.4mm}
\small{
Ma passion pour les Mathématiques couplée à un désir insatiable de comprendre l'existence m'a amené à recevoir divers et riches enseignements à travers differents pays, à travailler dans plusieurs organisations et à faire la rencontre de magnifiques personnes. Aujourd'hui, étudiant en Master de Statistiques Appliquées et Data Science, je souhaite continuer sur cette voie en exerçant plus tard la profession de chercheur en IA avec une spécialité en modèles graphiques probabilistes, Théorie de la fiabilité des Systèmes et Apprentissage profond.


}
\vspace{-3mm}
% ------------------ FORMATION -----------------------------
\section{\textbf{FORMATION}}
\vspace{-1mm}
\resumeSubHeadingListStart
\resumeSubheading
{Master 2, Statistiques Appliquées et Data Science (MASERATI)}{Septembre 2024 - En cours}
{Université Paris-Est Créteil}{Créteil, France}
\resumeItemListStart
\item Cours : Séries temporelles, Econométrie Avancée, Apprentissage Statistique, Big Data, Python,SAS
\resumeItemListEnd
\vspace{-1.5mm}

\resumeSubheading
{Bachelor, Statistiques Appliquées Option Économie (Erasmus en Suisse)}{Janvier 2023 - Juillet 2023}
{Haute École Spécialisée de Suisse Occidentale}{Genève, Suisse}
\resumeItemListStart
\item Cours : Analyse numérique, Optimisation, Statistiques inférentielles avancées, Programmation et Réseaux.
\resumeItemListEnd
\vspace{-1.5mm}

\resumeSubheading
{Bachelor, Finance}{Septembre 2021 - Août 2024}
{Institut Supérieur du Commerce de Paris}{ Levallois-Perret, France}
\resumeItemListStart
\item Cours : Mathématiques Financières, Probabilité, Informatique, Risk Management, Gestion de projet
\resumeItemListEnd
\resumeSubHeadingListEnd
\vspace{-7mm}
% -------------- EXPERIENCES ----------------------
\section{\textbf{EXPÉRIENCES PROFESSIONNELLES}}
\vspace{-0.4mm}
    \resumeSubHeadingListStart
    \resumeSubheading
        {Data Science - Stage R\&D}{Janvier 2024 - Juin 2024}
        {COLAS - Centre R\&D}{Saint-Quentin-en-Yvelines, France}
        \resumeItemListStart
            \item Ce stage consistait à optimiser l'entretien des routes en développant des algorithmes capable de prédire la formation et la propagation des fissures sur les chaussées. Les algorithmes d'apprentissage automatique, d'apprentissage profond et de durée de vie sélectionnés à partir de l'état de l'art ont été élaborés sur Microsoft Azure et Databricks en Python. 
        \resumeItemListEnd
        
\vspace{-0.4mm}

    \resumeSubheading
        {Chargé d’étude Statistique}{Février 2023 - Juillet 2023}
        {HEG - Département de Statistiques}{Genève, Suisse}
        \resumeItemListStart
            \item L'objectif de cette enquête statistique était d'analyser la relation entre l'évolution des loyers à Annemasse et à Genève. Pour cela, les tests statistiques, la méthode d'allocation optimale de Neyman et les modèles de différences ont été déployés en R.
        \resumeItemListEnd

\vspace{-0.4mm}

    \resumeSubheading
        {Assistant chargé d'étude marketing - Stage}{Janvier 2022 - Mars 2022}
        {Laboratoire CORELAB}{Levallois-Perret, France}        \resumeItemListStart
            \item L'objectif de ce stage était de réaliser une veille technologique sur les procédés à froid utilisés dans l'industrie cosmétique pendant la période post-covid. Cette veille comprenait l'étude des tendances, des brevets et des concurrents du laboratoire.
        \resumeItemListEnd

    \resumeSubHeadingListEnd
\vspace{-7mm}
% ----------------- TRAVAUX ACADEMIQUES --------------------------
\section{\textbf{TRAVAUX ACADEMIQUES}}
\vspace{-0.4mm}

    \resumeSubHeadingListStart
    \resumeSubheading
        {Mémoire Master 1}{Septembre 2024 - Juillet 2025}
        {Université Paris-Est Créteil}{Créteil, France}
        \resumeItemListStart
            \item Ce mémoire portait sur le développement d'un Réseau Bayésien Dynamique (Gaussien) pour prédire la consommation d'énergie en utilisant les données à haute fréquence collectées à partir des capteurs intelligents (Smart Meters). 
        \resumeItemListEnd
    \resumeSubHeadingListEnd
\vspace{-6mm}
% ----------------- COMPETENCES MATHEMATIQUES --------------------------
\section{\textbf{COMPÉTENCES MATHÉMATIQUES}}
\vspace{-0.2mm}
\small{
\textbf{Statistiques \& Econométrie :} Descriptive, Inférentielle, Régression généralisée, Probit, Logit, Panel. \\
\textbf{Machine \& Deep Learning :} Arbre de décision, RF, XGBoost, SVM, MLP, CNN, ANN. \\
\textbf{Séries Temporelles \& Simulation :} ARIMA, GARCH, VAR, MCMC. \\
\textbf{Optimisation \& Durée de vie :} Descente Gradient, Solveur, Estimateur KM, Modèle de Cox. \\
}
\textbf{Analyse Bayésienne \& NLP :} Réseau Bayésien Dynamique, tokenization, stopwords, lemmatisation. \\
}
\vspace{-7mm}
% ----------------- COMPETENCES INFORMATIQUES --------------------------
\section{\textbf{COMPÉTENCES INFORMATIQUES}}
\vspace{-0.2mm}
\small{
\textbf{Langages :} Python, R, SAS, SQL, C++, VBA, LaTeX. \\
\textbf{Librairies :} Pandas, Matplotlib, Scikit-learn, Tensorflow, Tidyverse, lubridate, Caffe, Etape data, Macrofunction. \\
\textbf{Bases de données et Gestion de projet :} GitHub/GitLab, Microsoft Azure, Databricks, Google Colab, Powershell. \\
\textbf{Bureautique et Visualisation :}  Word, Notion, Overleaf, Tableau Software, Tableau Software.
\vspace{-2mm}


% ----------------- LANGUES --------------------------
\section{\textbf{LANGUES}}
\vspace{-0.4mm}
\small{
\textbf{Français :} C2, Natif,  \textbf{Anglais :} B2, Courant, \textbf{Allemand :} A2, Débutant 
}

% ----------------- FIN DU DOCUMENT --------------------------
\end{document}
